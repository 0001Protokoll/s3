%\input{usepackage.tex}

%\begin{document}

\section{Einleitung}
Befasst man sich mit der Spektroskopie, so betrachtet man im wesentlichen die Beschreibung von Wechselwirkung elektromagnetischer Strahlung mit Materie. Dieses Teilgebiet der Physik findet in jeder Naturwissenschaft, bis hin zur Medizin, große Anwendungsmöglichkeiten, z.B als Analysemethode. In der organischen und anorganischen Chemie kann unter anderem durch geeignete Messungen die Reinheit eines Stoffes überprüft werden. Neben der Anwendung ist die Theorie der Spektroskopie, insbesondere der Molekülspektroskopie, für das Verständnis von photoneninduziertem Energietransportprozesse in einem chemischen System essenziell. Viele physikochemische Probleme konnten erst durch eine Betrachtung über die Spektroskopie gelöst werden, was ein tiefgreifendes Verständnis von Spektroskopie motivieren kann.




%\end{document}





