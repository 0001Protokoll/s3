%%\documentclass[a4paper, 12pt]{scrreprt}

\documentclass[a4paper, 12pt]{scrartcl}
%usepackage[german]{babel}
\usepackage{microtype}
%\usepackage{amsmath}
%usepackage{color}
\usepackage[utf8]{inputenc}
\usepackage[T1]{fontenc}
\usepackage{wrapfig}
\usepackage{lipsum}% Dummy-Text
\usepackage{multicol}
\usepackage{alltt}
%%%%%%%%%%%%bis hierhin alle nötigen userpackage
\usepackage{tabularx}
\usepackage[utf8]{inputenc}
\usepackage{amsmath}
\usepackage{amsfonts}
\usepackage{amssymb}

%\usepackage{wrapfig}
\usepackage[ngerman]{babel}
\usepackage[left=25mm,top=25mm,right=25mm,bottom=25mm]{geometry}
%\usepackage{floatrow}
\setlength{\parindent}{0em}
\usepackage[font=footnotesize,labelfont=bf]{caption}
\numberwithin{figure}{section}
\numberwithin{table}{section}
\usepackage{subcaption}
\usepackage{float}
\usepackage{url}
%\usepackage{fancyhdr}
\usepackage{array}
\usepackage{geometry}
%\usepackage[nottoc,numbib]{tocbibind}
\usepackage[pdfpagelabels=true]{hyperref}
\usepackage[font=footnotesize,labelfont=bf]{caption}
\usepackage[T1]{fontenc}
\usepackage {palatino}
%\usepackage[numbers,super]{natbib}
%\usepackage{textcomp}
\usepackage[version=4]{mhchem}
\usepackage{subcaption}
\captionsetup{format=plain}
\usepackage[nomessages]{fp}
\usepackage{siunitx}
\sisetup{exponent-product = \cdot, output-product = \cdot}
\usepackage{hyperref}
\usepackage{longtable}
\newcolumntype{L}[1]{>{\raggedright\arraybackslash}p{#1}} % linksbündig mit Breitenangabe
\newcolumntype{C}[1]{>{\centering\arraybackslash}p{#1}} % zentriert mit Breitenangabe
\newcolumntype{R}[1]{>{\raggedleft\arraybackslash}p{#1}} % rechtsbündig mit Breitenangabe
\usepackage{booktabs}
\renewcommand*{\doublerulesep}{1ex}
\usepackage{graphicx}


\usepackage[backend=bibtex, style=chem-angew, backref=none, backrefstyle=all+]{biblatex}
\bibliography{Literatur.bib}
\defbibheading{head}{\section{Literatur}\label{sec:Lit}} 
\let\cite=\supercite 
%\begin{document}
\section{Zusammenfassung}
Es wurde ein Rotationschwingungsspektrum von CO mittls Fourier-Transform-Infrarot-Spektrometer aufgenommen. Durch das Auftragen der Wellenzahl der Peaks gegen die Laufzahl $m$ konnte die Summe und Differenz der Rotationskonstanten vom Grund- und angeregten Zustand und die Position der Nullücke bestimmt werden, aus denen die entsprechenden Werte der beiden Zustände sowie die Rotationskonstante im Minimum der Potentialkurve und die drei dazugehörigen Abstände berechnet wurden. Sämmtliche Werte sind in Tablle \ref{Zusammenfassung} zusammengefasst.

\begin{table}
\label{Zusammenfassung}
	\caption{Zusammenfassung der berechneten Konstanten aus dem Rotationsschwingungspektrum von CO.}
\begin{tabular}{C{0.2\linewidth}|C{0.2\linewidth}|C{0.2\linewidth}|C{0.2\linewidth}}
Konstante             & Einheit                  & Wert                           & Literatur \\ \hline
$B_0$                  & [$cm^{-1}$]          & $1.9167 \pm 0.0006$	& $1.68172$   \\
$B_1$                  & [$cm^{-1}$]          & $1.896 \pm 0.002$   	& $1.66268$   \\
$B_e $                 & [$cm^{-1}$]          & $1.927 \pm 0.002$   	& $1.69124$   \\
$\alpha$		& [$cm^{-1}$]       & $0.081 \pm 0.003$   	& $0.01904$   \\
$r_e$                  & $\SI{}{[\angstrom]}$		& $1.129 \pm 0.006 $  	& $1.206$     \\
$r_0$                  & $\SI{}{[\angstrom]} $		& $1.132 \pm 0.004$  	 & $1.209$     \\
$r_1$                  & $\SI{}{[\angstrom]}$ 		& $1.138 \pm 0.004$  	 & $1.216$    \\
$\nu_0$	&[$cm^{-1}$]          & $2143.40 \pm 0.03$&
\end{tabular}
\end{table}


%\end{document}


